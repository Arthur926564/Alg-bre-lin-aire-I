\lecture{12}{2024-10-17}{Générateur}{}


Soit $W$ un sous espace vectoriel $V$.
\begin{itemize}
    \item Le sous espace $W = Vect(v_1, \dots, v_k)$ est le sous espace engendré par les vecteurs $v_1, \dots, v_k$.
    \item Les vecteurs $v_1, \dots, v_k$ sont les générateurs de $W$
    \item L'ensemble $\{v_1, \dots, v_k\}$ forme une \textbf{partie génératrice}de $W$.
\end{itemize}
\begin{exemple}
    Il existe en général plusieurs parties génératrices:
    \[Vect\]
\end{exemple}
\paragraph{Partie libres}
Soit $W$ un sous-espace vectoriel de $V$.
\begin{itemize}
    \item Les vecteurs $v_1, \dots, v_k$ de $W$ sont linéairement indépendants si la seul combinaison linéaire $\alpha_1v_1 + \cdots + \alpha_kv_k$ qui donne le vecteur nul est la combinaison linéaire triviale:
    \[\alpha_1 = \cdots = \alpha_k = 0\]

    \item On dit que l'ensemble $\{v_1, \dots, v_k\}$ est une \textcolor{red}{partie libre} de $W$.
\end{itemize}


\paragraph{Bases canoniques}

\subparagraph{Le cas de $\mathbb{R}^n$}
La base canonique est 
\begin{formule}

\[
\mathcal{C}an = (\vec{e}_1, \vec{e}_2, \ldots, \vec{e}_n)
\]
\end{formule}
C'est une base car nous avons vu que tout vecteur de \R$^n$ s'écrit comme combinaison linéaire des $\vec{e}_i$.
\\
\subparagraph{Le cas de $P_n$}
La base canonique est 
\begin{formule}
    

\[
\mathcal{C}an = (1, t, t^2, \ldots, t^n)
\]
\end{formule}
Ici aussi tout vecteur de $\mathbb{P}_n$, i.e. tout polynôme de degré $\leq$ n s'écrit comme combinaison linéaire de ces monômes $t^i$, car un tel polynôme est de la forme $a_0\cdot 1 + a_1\cdot t + \dots + a_n\cdot t^n$.
\subparagraph{Le cas de $M_{m \times n}(\mathbb{R})$}
La base canonique est 
\begin{formule}
\[
\mathcal{C}an = (e_{11}, \ldots, e_{1n}, e_{21}, \ldots, e_{m1}, \ldots, e_{mn})
\]
\end{formule}
où $e_{ij}$ est la matrice constituée de zéros, sauf le coefficient $(i, j)$ qui vaut 1. Dans $M_{3\times 2}$(\R), la base canonique est donnée dans cet ordre :
\\
\[e_{11} = \begin{pmatrix}
    1 & 0\\
    0 & 0 \\
    0 & 0
\end{pmatrix}, e_{12} = \begin{pmatrix}
    0 & 1 \\ 0 & 0 \\ 0 & 0
\end{pmatrix}, e_{22} = \begin{pmatrix}
    0 & 0 \\ 1 & 0 \\ 0 & 0
\end{pmatrix} \dots, e_{32} = \begin{pmatrix}
    0 & 0 \\ 0 & 0 \\ 0 & 1
\end{pmatrix} \]

\subparagraph{Exemple}
Soit $W$ le plan dans \R$^3$ donné par l'équation $x + y + z = 0$. L'inconnue $x$ est principale, les inconnues, $y, z$ sont secondaires et seront nos paramètres.
\[x = -y -z\]
Les vecteurs $\vec{b_1} = \begin{pmatrix}
    -1 \\ 1 \\ 0
\end{pmatrix}$ et $\vec{b_2} =  \begin{pmatrix}
    -1 \\ 0  \\1
\end{pmatrix}$ forment une base $\beta = (\vec{b_1}, \vec{b_2}$) de $W$.
\begin{framedremark}
    Il n'y a pas de base canonique dans $W$. Nous avons fait des choix de paramètres, Il y a pas une "\textit{norme}" pour la base canonique ici.
\end{framedremark}
\begin{exemple}
    Si on considère $x, y$ comme inconnues libres, $z$ en principales, alors on trouve une autre base:
    \[c = \begin{pmatrix}
        
    \begin{pmatrix}
        1 \\ 0 \\ -1
    \end{pmatrix}, & \begin{pmatrix}
        0 \\ 1 \\ -1
    \end{pmatrix}\end{pmatrix}\]
\end{exemple}
\subparagraph{Théroème de la base extraite}
Soit $\{v_1, \dots, v_k\}$ une famille de vecteurs qui engendrent $V$.
\begin{theoreme}
    \begin{itemize}
        \item Si l'un des vecteurs $v_i$ est combinaison linéaire des autres, alors la famille obtenue en supprimant $v_i$ engendre encore $V$.
        \item Si $V \neq \{0\}$, il existe une sous-famille de $\{v_1, \dots, v_k\}$ qui forme une base de $V$
    \end{itemize}
\end{theoreme}
\subparagraph{Preuve} \textbf{(A)} pour $i = k$. On suppose que
\begin{formule}
    \[v_k = \alpha_1v_i + \dots + \alpha_{k-1}v_{k-1}\]
\end{formule}
Puisque la famille $\{v_1, \dots, v_k\}$ engendre $V$, tout vecteur $v \in V$ est combinaison linéaire de $v_i$.
\begin{align*}
v &= \beta_1 v_1 + \cdots + \beta_{k-1} v_{k-1} + \beta_k v_k \\
  &= \beta_1 v_1 + \cdots + \beta_{k-1} v_{k-1} + \beta_k (\alpha_1 v_1 + \cdots + \alpha_{k-1} v_{k-1}) \\
  &= (\beta_1 + \beta_k \alpha_1) v_1 + \cdots + (\beta_{k-1} + \beta_k \alpha_{k-1}) v_{k-1}
\end{align*}
Nous avons montré que $v$ est combinaison linéaire de $v_1, \dots, v_{k-1}$.
\\
\textbf{(B)} Si la famille est libre on arrête tout, sinon il suffit de prendre notre point (A) et d'enlever un générateur $v_i$ et en enlevant un denouveau et un autre et un autre... jusqu'à que la famille soit libre. Le processus s'arrête puisque le nombre de vecteur au départ est \textbf{fini}.









\subparagraph{Combinéaisons linéaires d'une base}
Soit $V$ un espace vectoriel et $\mathcal{B} = (e_1 + \dots, e_n)$ une base.
\begin{theoreme}
    Tout vecteur $x$ de $V$ s'écrit de manière unique comme combinaison linéaire $x = x_1e_1 + \dots + x_ne_n$, pour des nombres réels $x_1, \dots, x_n$.
\end{theoreme}
\textbf{Existence} Une base est un \textcolor{red}{système de générateurs}!
\\
\textbf{Unicité} Si $x_1e_1 + \dots + x_ne_n = y_1e_1 + \dots y_ne_n$ alors, 
\[(x_1 - y_1)e_1 + \dots +(x_n - y_n)e_n = 0\]
Une base est \textcolor{red}{libre}, ainsi $x_1 = y_1, \dots, x_n = y_n$
\paragraph{Coordonnées}
\begin{definition}
    Les composantes ou coordonnées d'un vecteur $x$ dans la base $\mathcal{B}$ sont les coefficients réels $x_1, \dots, x_n$ tel que
    \[x = x_1e_1 + \dots + x_ne_n\]
\end{definition}
J'affirme que $
B = \left(
\begin{pmatrix}
1 \\
0 \\
-1
\end{pmatrix},
\begin{pmatrix}
1 \\
-1 \\
0
\end{pmatrix},
\begin{pmatrix}
0 \\
1 \\
1
\end{pmatrix}
\right)
$ est une base de \(\mathbb{R}^3\). En effet, $\begin{pmatrix}
    1 & 1 & 0 \\
    0 & -1 & 1 \\
    -1 & 0 & 1
\end{pmatrix}$ est inversible. Le vecteur $\vec{u} = \begin{pmatrix}
    1 \\ 2 \\ 3
\end{pmatrix}$ de \R$^3$ s'exprime comme combinaison linéaire des $\vec{b}_i$ et on cherche $(\vec{u})_\mathcal{B}$. On doit résoudre :

\[\left ( \begin{array}{ccc|c}
    1 & 1 & 0 & 1 \\
    0 & -1 & 1 & 2 \\
    -1 & 0 & 1 & 3
\end{array} \right) \to \left (\begin{array}{ccc|c}
    1 & 0 & 0 & 0 \\
    0 & 1 & 0 & 1 \\
    0 & 0 & 1 & 3 
\end{array}\right )\]

La solution est :
$(\vec{u})_\mathcal{B} =  \begin{pmatrix}
    0 \\ 1 \\ 3
\end{pmatrix}$ et $(\vec{u})_{\mathcal{C}
an} = \begin{pmatrix}
    1 \\ 2 \\ 3
\end{pmatrix}$
\subparagraph{Comparaison avec \R$^n$}
\begin{definition}
    Une application linéaire bijective et appelée \textcolor{red}{isomorphisme}
\end{definition}
Un isomorphisme permet d'identifier la source et le but de cette application linéaire $T : V \to W$. Les éléments de $V$ et $W$ se correspondent parfaitement, et les opérations de somme et d'action aussi.
\begin{theoreme}
    Soit $V$ une espace vectoriel et $\mathcal{B}$ une base de $n$ vecteurs. \\
    L'application $T : V \to \mathbb{R}^n$ définie par $T(x) = (x)_{\mathcal{B}}$ est un isomorphisme.
\end{theoreme}
\begin{dem}
    Nous devons prouver quatre points, les deux premiers pour montrer que $T$ est linéaire, les deux autres pour établir l'injectivité, et enfin la surjectivité.
    \begin{itemize}
        \item $T(\lambda v) = \lambda T(v)$ pour tout $\lambda \in$ \R et tout $v \in V$;
        \item $T(v + w) = T(v) + T(w)$ pour tous $v, w \in V$;
        \item $T(v) = \vec{0} \implies v = 0$ (critère d'injectivité);
        \item Pour tout $\vec{b} \in$ \R$^n$ il existe $v \in V$ tel que $T(v) = \vec{b}$

    \end{itemize}
    Soit donc $\lambda \in$ \R et $v, w \in V$ que nous écrivons de manière unique! comme $v = x_1e_1 + \dots + x_ne_n$ et $w = y_1e_1 + \dots + y_ne_n$.
\end{dem}
Prouve que le premier point:

\begin{align*}
T(\lambda v) &= T ( \lambda(v_1b_1 + \dots + v_nb_n)) \\
&= T(\lambda v_1 b_1 + \dots + \lambda v_nb_n) \\
&= T\begin{pmatrix}
    \lambda v_1 \\
    .\\\\. \\ \lambda v_n
\end{pmatrix} = \lambda \begin{pmatrix}
    v_1 \\ \\. \\. \\v_n
\end{pmatrix} = \lambda T(v)
\end{align*}

\begin{exemple}
    \begin{itemize}
        \item $T : M_{x\times2}(\mathbb{R}) \to $ \R$^4$ est un isomorphisme. Pour la base canonique $\mathcal{C}an = (e_{11}, e_{12}, e_{21}, e_{22})$, on a:
\[T\begin{pmatrix}
    a & b \\ c & d
\end{pmatrix} = \left ( \begin{pmatrix}
    a & b \\ c & d
\end{pmatrix}\right )_{\mathbf{C}can} = \begin{pmatrix}
    a \\ b \\ c \\ d
\end{pmatrix}\]
        \item $T : \mathbb{P}_3 \to $\R$^4$ est un isomorphisme. Pour la base canonique $\mathcal{C}an = (1, t, t^2, t^3)$ on a
\[T(a + bt + ct^2 + dt^3) = \begin{pmatrix}
    a \\ b \\ c \\ d
\end{pmatrix}\]
\item Soit $W$ le plan dans \R$^3$ donné par l'équation $x + y + z = 0$ et $\mathcal{B}$ la base formée des vecteurs $\vec{b}_1 = \begin{pmatrix}
    -1 \\ 1 \\ 0
\end{pmatrix}$ et $\vec{b}_2 =  \begin{pmatrix}
    -1 \\ 0 \\ 1
\end{pmatrix}$
On a l'application $T : W \to \mathbb{R}^2$
\[w \to \begin{pmatrix}
    a \\ b
\end{pmatrix} \text{ si } w = a\cdot\vec{b_1} + b\cdot\vec{b}_2\]
\end{itemize}

\end{exemple}


\subparagraph{Base et coordonnées}
\begin{definition}[pivots]
    La famille $(p, q, r)$ est une base de $\mathbb{P}^2$ si et seulement si la matrice carrée $A$ a trois pivots. Où $A$ est la matrice d'application avec les vecteurs $(p, q, r)$ fourni dans la base canonique.
\end{definition}

\paragraph{Cardinalité d'une base}
Pour généraliser ce qu'on a dit avant:
\begin{itemize}
    \item $V$ est un espace vectoriel
    \item $\mathcal{B} = (b_1, \dots, b_n)$ une base $V$
    \item $\mathcal{C} = (c_1, \dots, c_m)$ une famille ordonnée de vecteurs de $V$.
\end{itemize}

\begin{theoreme}
    La famille ordonnée de $\mathcal{C}$ est une base de $V$ si et seulement si la matrice $A = ((c_1)_{\mathcal{B}}, \dots, (c_m)_{\mathcal{B}})$ a un pivot dans chaque ligne et chaque colonne.
\end{theoreme}
\begin{framedremark}
    En particulier $A$ est une matrice carrée ($m = n$) et inversible
\end{framedremark}
\begin{theoreme}
    Deux bases de $V$ ont le même nombre d'éléments
\end{theoreme}
\begin{corollaire}
Si $V$ admet une base de $n$ vecteurs, alors une famille $\{v_1, \dots, v_k\}$ de vecteurs de $V$ avec $k > n$ est liée.
\end{corollaire}
\begin{definition}
    Soit $V$ un espacevectoriel et $\mathcal{B} = (e_1, \dots, e_n)$ une base. La \textcolor{red}{dimension} de $V$ est $n$. On note $dim V = n$.
\end{definition}

\begin{itemize}
    \item $dim \mathbb{R}^n = n$
    \item $dim \mathbb{P}_n = n+1$
    \item $M{m\times n}(\mathbb{R}) = mn$
\end{itemize}